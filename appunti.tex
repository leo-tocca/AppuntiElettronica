% Options for packages loaded elsewhere
\PassOptionsToPackage{unicode}{hyperref}
\PassOptionsToPackage{hyphens}{url}
%
\documentclass[
]{article}
\usepackage{amsmath,amssymb}
\usepackage{iftex}
\ifPDFTeX
  \usepackage[T1]{fontenc}
  \usepackage[utf8]{inputenc}
  \usepackage{textcomp} % provide euro and other symbols
\else % if luatex or xetex
  \usepackage{unicode-math} % this also loads fontspec
  \defaultfontfeatures{Scale=MatchLowercase}
  \defaultfontfeatures[\rmfamily]{Ligatures=TeX,Scale=1}
\fi
\usepackage{lmodern}
\ifPDFTeX\else
  % xetex/luatex font selection
\fi
% Use upquote if available, for straight quotes in verbatim environments
\IfFileExists{upquote.sty}{\usepackage{upquote}}{}
\IfFileExists{microtype.sty}{% use microtype if available
  \usepackage[]{microtype}
  \UseMicrotypeSet[protrusion]{basicmath} % disable protrusion for tt fonts
}{}
\makeatletter
\@ifundefined{KOMAClassName}{% if non-KOMA class
  \IfFileExists{parskip.sty}{%
    \usepackage{parskip}
  }{% else
    \setlength{\parindent}{0pt}
    \setlength{\parskip}{6pt plus 2pt minus 1pt}}
}{% if KOMA class
  \KOMAoptions{parskip=half}}
\makeatother
\usepackage{xcolor}
\setlength{\emergencystretch}{3em} % prevent overfull lines
\providecommand{\tightlist}{%
  \setlength{\itemsep}{0pt}\setlength{\parskip}{0pt}}
\setcounter{secnumdepth}{-\maxdimen} % remove section numbering
\usepackage{circuitikz}
\usepackage{geometry}
    \geometry{
        a4paper,
        total={170mm,257mm},
        left=20mm,
        top=20mm,
    }
\ifLuaTeX
  \usepackage{selnolig}  % disable illegal ligatures
\fi
\usepackage{bookmark}
\IfFileExists{xurl.sty}{\usepackage{xurl}}{} % add URL line breaks if available
\urlstyle{same}
\hypersetup{
  pdftitle={Appunti Elettronica generale},
  pdfsubject={Elettronica},
  hidelinks,
  pdfcreator={LaTeX via pandoc}}

\title{Appunti Elettronica generale}
\author{}
\date{2024-04-12}

\begin{document}
\maketitle

\section{Semiconduttori}\label{semiconduttori}

Sono, come suggerisce il nome, materiali in cui il flusso di corrente
\emph{non è libero} (non è un conduttore), ma è \textbf{presente} (non è
un'isolante).

In particolare, conducono in particolari situazioni. Quali sono però i
materiali con queste condizioni? - Silicio (Si), Germanio (Ge) (Carbonio
(C), ma composto) - GaAs, GaN (Gallio-Arsenico/Azoto) In generale sono
gli elementi della \(4°\) colonna della tavola periodica o composti a
numero medio di elettroni liberi pari a 4.

Per far condurre un materiale va \emph{``drogato''}, ossia aggiungere,
in piccole dosi, materiali della \(5°\) colonna (drogaggio di tipo N),
che cedono \emph{elettroni}, o elementi della \(3°\) colonna (tipo P),
che acquistano elettroni. La qualità del semiconduttore è influenzata
dal materiale usato (per esempio Ge è meglio del Si, ma è più raro), che
è a sua volta influenzato dal goal\footnote{(penso voglia dire
  ``obiettivo perseguito'')}(elettronica digitale usa Si, l'elettronica
di potenza il GaN o SiC).

Vediamo ora degli elementi in silicio.

\subsection{Giunzione p-n}\label{giunzione-p-n}

Il drogaggio di un materiale è quantificato con le grandezze \[
N_A = \frac{\# acceptors}{vol. unit} \text{ e } N_d=\frac{\# donors}{vol. unit}
\] dove \(N_a\) è tipo p:`positivo', mentre \(N_d\) è di tipo
n:`negativo'.

\begin{quote}
Memo: il silicio puro ha una struttura cristallina matriciale, che
blocca il passaggio di carica.
\end{quote}

Collegando un blocco drogato tipo ? ed uno tipo ? abbiamo (ideale)

\begin{quote}
Inserire immagine
\end{quote}

L'abbondanza di lacune in \(p\) è considerabile come una carenza di
elettroni, di cui \(n\) \emph{abbonda}. Ciò genera una
\textbf{migrazione} di elettroni da n verso p.

\begin{quote}
Inserire immagine
\end{quote}

Tale fenomeno carica in modo \emph{positivo} n (meno elettroni), e in
modo \emph{negativo} p (più elettroni).

Tali cariche generano dei campi elettrici (positivo su n e negativo su
p), che impedisce un ulteriore passaggio di carica, si ha allora un
\textbf{equilibrio}.

\begin{quote}
Inserire immagine
\end{quote}

Nel punto di contatto si crea così una zona in cui tutte le lacune (di
cosa?) sono state riempite, e tutti gli elettroni extra di ? ceduti.
Tale zona è detta \textbf{depletion layer} (regione di svuotamento a
carica spaziale).

\begin{quote}
Inserire immagine
\end{quote}

In genere il dp non è simmetrico: deve valere: \[
x_p N_A = x_n N_D
\] Ricordando che \(E=\int Q\), e che \(V=\int E\), si vede come il
potenziale \textbf{impedisca} il moto \(p\rightarrow n\) (delle lacune)
e \(n\rightarrow p\) (degli elettroni).

Il simbolo circuitale della giunzione pn, detta \textbf{diodo} è

\begin{center}
\begin{circuitikz}
\draw (0,0) to[diode] (2,0); 
\end{circuitikz}
\end{center}

dove a sinistra abbiamo un \textbf{anodo} A (dal greco \emph{salita}), e
a destra un \textbf{catodo} (dal greco \emph{discesa}).

\end{document}
